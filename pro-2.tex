\section*{(2)}
\begin{figure}[h]
  \centering
  \begin{tikzpicture}[
    > = latex,
    dot/.style = {draw,fill,circle,inner sep=1pt},
    arrow inside/.style = {postaction=decorate,decoration={markings,mark=at position .55 with \arrow{>}}}
    ]
    %\draw[<->] (0,6) node[above right] {$P$} |- (6,0) node[right] {$V$};
    %\draw[->] (0,0) node[above] {$P$} %|- (6,0) node[right] {$V$};
    \draw[->] (0,0) -- (8,0) node[right]{$X$};
    \draw[->] (0,0) -- (0,5) node[left]{$T$};
    \node[dot,label={right:$3$}] (@a) at (6,4) {};
    \node[dot,label={below left:$1$}] (@b) at (2.5,4) {};
    \node[dot,label={below left:$2$}] (@c) at (6,1.5) {};

  
    \draw[arrow inside] (@a) to[ left=20] (@b);
    \draw[arrow inside] (@b) to[looseness=.95,bend right=20] (@c);
    \draw[arrow inside] (@c) to[right=20] (@a);
 
    \draw[dashed] (@a) to [left] (0,4);
    \draw[dashed] (@c) to [left] (0,1.5);
    \draw (-.98,4.0) node {$T'$};
    \draw (-.98,1.5) node {$T$};
    \draw[dashed] (@b) to [left] (2.5,0);
    \draw[dashed] (@c) to [left] (6,0);
    \draw (6.0,-.98) node {$X_1$};
    \draw (2.5,-.98) node {$X_2$};
  \end{tikzpicture}
  \caption{等温サイクル}
  \label{carnot}
  \end{figure}
図2のような熱浴$T'$を用いた等温サイクルを考える.等温サイクルより,Kelvinの原理から
\begin{equation}\label{Kelvinの原理}
  W(1 \rightarrow 2) + W(3\rightarrow 1) \leq 0 \tag{2.1}
\end{equation}
過程$1 \rightarrow 2$はaq過程であるから.
\[
  W(1 \rightarrow 2) = U(1) - U(2)
\]
過程$3 \rightarrow 1$はiq過程であるから
\begin{align*}
  W(3\rightarrow 1) 
  &= F(3) - F(1)\\
  &= U(3) - T'S(3) - U(1) + T'S(1) \\
  &=- T'S(3) + T'S(1) 
\end{align*}
式(\ref{Kelvinの原理})より
\[
  U(1) - U(2) - T'S(3) + T'S(1) \leq 0
\]
$U(1) = U(3)$より
\[
  U(3) - U(2) - T'S(3) + T'S(1) \leq 0
\]
\noindent (1)の結果より$S(1)=S(2)$より
\[
  \frac{U(T',X_1)-U(T,X_1)}{T'} \leq S(T',X_1) - S(T,X_1)
\]
一方,UはTの増加関数だから$\frac{U(T',X_1)-U(T,X_1)}{T'} \geq 0$より
\[(T',X_1) - S(T,X_1)\geq 0\]
従って,エントロピーはTの増加関数である.

