\documentclass[dvipdfmx,a4paper]{jsarticle}

\usepackage{amsmath} 
\usepackage{lipsum}
\usepackage{xpatch}
\usepackage{tikz}
\usepackage{color}
\usepackage{latexsym}
\usepackage{amssymb}

\usetikzlibrary{arrows,decorations.markings}

\title{熱物理2023年度期末 過去問解答}
\author{著者:
Y I\\Keio University \\department of applied physics and physico information}

\begin{document}

\maketitle


\begin{abstract}
  2023年度熱物理期末の解答をここに記します。これは私の \LaTeX の練習のため作られたPDFであり、
  解答の精度については責任を持てません。この書類の著作権については放棄をしません。しかし、この書類の所有者は許可なく自由に
  この書類を、慶應義塾に所属している塾生に譲渡することを認めます。
\end{abstract}

\input pro-1.tex
\input pro-2.tex
\input pro-3.tex
\color{red}
以降の設問は筆者が確認できる限り、数値や出題順が違えど2020年から2023年連続でまったく同じ
出題がされています
\color{black}
\input pro-4.tex
\input pro-5.tex
\input pro-6.tex
\input pro-7.tex


\end{document}
